\documentclass[letterpaper]{article}
\usepackage{natbib,alifexi}
\usepackage[utf8]{inputenc}
\usepackage[frenchb]{babel}

\title{Time stretching en temps réel pour le live coding}
\author{Abdeselam El-Haman Abdeselam$^{1}$\\
Superviseur : Bernard Fortz
\mbox{}\\\\
$^1$Université Libre de Bruxelles \\
aelhaman@ulb.ac.be}


\begin{document}
\maketitle

\begin{abstract}
  Overtone est une libraire en Clojure qui est utilisée pour faire du
  Live Coding (l'art de programmer en \og vif \fg{}). Une des techniques les
  plus utilisées dans le domaine de la musique synthétisée est le
  time-stretching, qui consiste à rallonger ou rétrecir une pièce musicale
  sans changer sa tonalité. Le time-stretching est intéressant dans
  le live-coding lorsqu'on peut modifier les paramétres de celui-ci
  en temps réel. Dans cet article 2 méthodes de time-stretching avec des
  approches différentes seront analysées, comparées et utilisées en temps
  réel dans Overtone.

\end{abstract}

\section{Introduction}
Parler du son: C'est quoi le son. Fréquences.

C'est quoi problème time stretching?
À quoi est-il dû?

\section{État de l'art}
Techniques, un peu d'histoire?

\subsection{Traitement du temps}
Parler de SOLA, PSOLA...
\subsection{Traitement des fréquences}
Parler du Vocodeur de phases, comment c'est utilisé etc...
\section{Implémentation}
Parler d'Overtone, Supercollider...
UGENs qu'on va utiliser: PVRecordBuf, PVPlayBuf..


\footnotesize
\bibliographystyle{apalike}
\bibliography{rapport}


\end{document}
